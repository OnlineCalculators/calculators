% ****** Start of file apssamp.tex ******
%
%   This file is part of the APS files in the REVTeX 4.1 distribution.
%   Version 4.1 of REVTeX, October 2009
%
%   Copyright (c) 2009 The American Physical Society.
%
%   See the REVTeX 4 README file for restrictions and more information.
%
% TeX'ing this file requires that you have AMS-LaTeX 2.0 installed
% as well as the rest of the prerequisites for REVTeX 4.1
%
% See the REVTeX 4 README file
% It also requires running BibTeX. The commands are as follows:
%
%  1)  latex apssamp.tex
%  2)  bibtex apssamp
%  3)  latex apssamp.tex
%  4)  latex apssamp.tex
%
\documentclass[%
 reprint,
%superscriptaddress,
%groupedaddress,
%unsortedaddress,
%runinaddress,
%frontmatterverbose, 
%preprint,
%showpacs,preprintnumbers,
%nofootinbib,
%nobibnotes,
%bibnotes,
 amsmath,
 amssymb,
 aps,
 prl,
%pra,
%prb,
%rmp,
%prstab,
%prstper,
%longbibliography,
%floatfix,
 lengthcheck,%
]{revtex4-1}

%\usepackage{epsfig}
% \usepackage{pst-grad} % For gradients
% \usepackage{pst-plot} % For axes


\usepackage{graphicx}% Include figure files
\usepackage{dcolumn}% Align table columns on decimal point
\usepackage{bm}% bold math
\usepackage{hyperref}% add hypertext capabilities
\usepackage{amssymb}
%\usepackage[mathlines]{lineno}% Enable numbering of text and display math
%\linenumbers\relax % Commence numbering lines

%\usepackage[showframe,%Uncomment any one of the following lines to test 
%%scale=0.7, marginratio={1:1, 2:3}, ignoreall,% default settings
%%text={7in,10in},centering,
%%margin=1.5in,
%%total={6.5in,8.75in}, top=1.2in, left=0.9in, includefoot,
%%height=10in,a5paper,hmargin={3cm,0.8in},
%]{geometry}

\begin{document}

\preprint{APS/123-QED}

\title{Calculation of the spin-dependent wavefunction in a magnetic sample, in the reflection geometry}% Force line breaks with \\

\author{Brian B. Maranville}
 \email{brian.maranville@nist.gov}
\affiliation{%
 NIST Center for Neutron Research\\
 100 Bureau Drive, Gaithersburg MD USA 20899
}%

\date{\today}% It is always \today, today,
             %  but any date may be explicitly specified

\begin{abstract}
The theory of polarized neutron reflectometry has long been worked out, and the 
amplitude of the reflected and transmitted waves (at large distances from the 
sample) can be easily calculated using
available computer code.  When using the one-dimensional reflectivity theory in
the special case of the distorted-wave Born approximation (DWBA), the amplitudes
of the wavefunctions are needed inside the sample as well, and in this paper
we describe the implementation of this calculation.
\end{abstract}

\pacs{Valid PACS appear here}% PACS, the Physics and Astronomy
                             % Classification Scheme.
%\keywords{Suggested keywords}%Use showkeys class option if keyword
                              %display desired
\maketitle

\section{Spin-polarized neutron reflectivity}
The theory of polarized neutron reflectivity is thoroughly described in the 
literature \cite{PNRMajkrzakChapter}.  The two coupled equations describing the
spin-dependent wavefunction are 

\begin{eqnarray}
    \left[ \frac{\partial^2}{\partial z^2} + \frac{Q^2}{4} - 4\pi\rho_{++}(z) \right ] \psi_{+}(z) 
    	- 4\pi\rho_{+-}(z) \psi_{-}(z) = 0 \nonumber \\
    \left[ \frac{\partial^2}{\partial z^2} + \frac{Q^2}{4} - 4\pi\rho_{--}(z) \right ] \psi_{-}(z) 
    	- 4\pi\rho_{-+}(z) \psi_{+}(z) = 0 \nonumber \\
\end{eqnarray}
where 
\begin{equation}
  \begin{pmatrix} 
    \rho_{++} & \rho_{+-} \\ 
    \rho_{-+} & \rho_{--} \\ 
  \end{pmatrix}
= \begin{pmatrix}
    Nb + Np_z & Np_x - iNp_y \\
    Np_x + iNp_y & Nb - Np_z \\
  \end{pmatrix}
\end{equation}
The $\hat z$ direction in the sample coordinate system is the surface normal, 
which is of course also the direction of the momentum transfer $\vec Q$.  
As described in \cite{PNRMajkrzakChapter}, the two coupled equations can be 
combined to give two fourth-order uncoupled equations, and if we use a 
solution of the form $\psi=\exp(Sz)$ the four roots of $S$ are
\begin{equation}
  \begin{array}{l}
  S_1 = \sqrt{4\pi(Nb + Np) - Q^2/4} \\
  S_2 = -S_1 \\
  S_3 = \sqrt{4\pi(Nb - Np) - Q^2/4} \\
  S_4 = -S_3 \\
  \end{array}
\end{equation}

and the total wavefunction is the combination of one that is spin-up and one
that is spin-down (in the sample reference frame.) 
\begin{equation}
  \begin{array}{l}
  \displaystyle \Psi(z) = \begin{pmatrix} \psi_{+}(z) \\ \psi_{-}(z) \end{pmatrix}\\
  \displaystyle \psi_{+}(z) = \sum_{j=1}^{4} C_j e^{S_j z} \\
  \displaystyle \psi_{-}(z) = \sum_{j=1}^{4} D_j e^{S_j z} \\
  \end{array}
\end{equation}

The coupling from the first equation leads to a relationship between the 
coefficients of the two sub-wavefunctions: 
\begin{equation}
  D_j = \mu_j C_j
\end{equation}
where 
\begin{equation}
\begin{array}{l}
  \mu_1 = \frac{1+\cos\theta_M+i\sin\theta_M\cos\phi_M - \sin\theta_M\sin\phi_M}
    {1+\cos\theta_M-i\sin\theta_M\cos\phi_M + \sin\theta_M\sin\phi_M} \\
  \mu_2 = \mu_1 \\
  \mu_3 = \frac{-1+\cos\theta_M+i\sin\theta_M\cos\phi_M - \sin\theta_M\sin\phi_M}
    {-1+\cos\theta_M-i\sin\theta_M\cos\phi_M + \sin\theta_M\sin\phi_M} \\
  \mu_4 = \mu_3 \\
\end{array}
\end{equation}
($\theta_M$ is the in-plane $\hat x$\nobreakdash-$\hat y$ rotation vector, zero at $\hat x$,
$\phi_M$ is the out-of-plane component along $\hat z$), 
but in the case where the $z$-component of the magnetization is held to be zero
(or constant through fronting/sample/backing) the angle $\phi_M$ is zero, and
$\mu$ is 
\begin{equation}
\begin{array}{l}
  \displaystyle \mu_1 = \mu_2 = e^{i\theta_M} \\
  \displaystyle \mu_3 = \mu_4 = -e^{i\theta_M} \\
\end{array}
\end{equation}

Then we can write a column vector for $\Psi(z)$ and $\Psi'(z)$ 
representing all the wavefunction 
terms that are continuous in the solution, as the second-order Schr\"odinger 
equation requires. 

\begin{equation}
\label{eq:cd_to_pz}
  %\begin{eqnarray}
  \begin{pmatrix}
    \psi_{+}(z) \\
    \psi_{-}(z) \\
    \psi'_{+}(z) \\
    \psi'_{-}(z) \\
  \end{pmatrix}
  = \chi
  \begin{pmatrix}
    C_1 \\
    C_2 \\
    C_3 \\
    C_4 \\
  \end{pmatrix}
\end{equation}
where 
\begin{equation}
  \chi = 
  \begin{pmatrix}
    e^{S_1 z} & e^{-S_1 z} & e^{S_3 z} & e^{-S_3 z} \\
    \mu e^{S_1 z} & \mu e^{-S_1 z} & -\mu e^{S_3 z} & -\mu e^{-S_3 z} \\
    S_1 e^{S_1 z} & -S_1 e^{-S_1 z} & S_3 e^{S_3 z} & -S_3e^{-S_3 z} \\
    \mu S_1 e^{S_1 z} & -\mu S_1 e^{-S_1 z} & -\mu S_3 e^{S_3 z} & \mu S_3 e^{-S_3 z} \\
  \end{pmatrix}
\end{equation}
or, separating out the z-dependent terms within a layer
\begin{equation}
  \chi = 
  \left(
  \begin{matrix}
    1 & 1 & 1 & 1 \\[0.3em]
    \mu & \mu & -\mu & -\mu \\[0.3em]
    S_1 & -S_1 & S_3 & -S_3 \\[0.3em]
    \mu S_1  & -\mu S_1 & -\mu S_3 & \mu S_3 \\[0.3em]
  \end{matrix}
  \right)_{\!\!\!\!l}
  \left(
  \begin{matrix}
    e^{S_1 z} & 0 & 0 & 0 \\[0.3em]
    0 & e^{-S_1 z} & 0 & 0 \\[0.3em]
    0 & 0 & e^{S_3 z} & 0 \\[0.3em]
    0 & 0 & 0 & e^{-S_3 z} \\[0.3em]
  \end{matrix}
  \right)
  %\end{eqnarray}
\end{equation}

%Noting that at $z=0$, the above gives:
%\begin{equation}
%  \label{eq:cd_to_p0}
%  \begin{pmatrix}
%    \psi_{+}(0) \\
%    \psi_{-}(0) \\
%    \psi'_{+}(0) \\
%    \psi'_{-}(0) \\
%  \end{pmatrix}
%  =
%  \begin{pmatrix}
%    1 & 1 & 1 & 1 \\
%    \mu & \mu & -\mu & -\mu \\
%    S_1 & -S_1 & S_3 & -S_3 \\
%    \mu S_1  & -\mu S_1 & -\mu S_3 & \mu S_3 \\
%  \end{pmatrix}
%  \begin{pmatrix}
%    C_1 \\
%    C_2 \\
%    C_3 \\
%    C_4 \\
%  \end{pmatrix}
%\end{equation}

The inverse of the first ($z$-independent) matrix above allows one to
calculate the $C_j$ in terms of $\Psi(z=0)$
\begin{equation}
\label{eq:p_to_cd}
\frac{1}{4}
  \begin{pmatrix}
    1 & \frac{1}{\mu} & \frac{1}{S_1} & \frac{1}{\mu S_1} \\[0.3em]
    1 & \frac{1}{\mu} & \frac{-1}{S_1} & \frac{-1}{\mu S_1} \\[0.3em]
    1 & \frac{-1}{\mu} & \frac{1}{S_3} & \frac{-1}{\mu S_3} \\[0.3em]
    1 & \frac{-1}{\mu} & \frac{-1}{S_3} & \frac{1}{\mu S_3} \\[0.3em]
  \end{pmatrix}
  \begin{pmatrix}
    \psi_{+}(0) \\[0.3em]
    \psi_{-}(0) \\[0.3em]
    \psi'_{+}(0) \\[0.3em]
    \psi'_{-}(0) \\[0.3em]
  \end{pmatrix}
  = 
  \begin{pmatrix}
    C_1  \\[0.3em]
    C_2  \\[0.3em]
    C_3  \\[0.3em]
    C_4  \\[0.3em]
  \end{pmatrix}
\end{equation}

Note that substituting the left side of Eq. \ref{eq:p_to_cd} into 
the right side of Eq. \ref{eq:cd_to_pz}
gives the transfer matrix: 
\begin{equation}
  \begin{pmatrix}
    \psi_{+}(z) \\
    \psi_{-}(z) \\
    \psi'_{+}(z) \\
    \psi'_{-}(z) \\
  \end{pmatrix}
  = 
  A
  \begin{pmatrix}
    \psi_{+}(0) \\
    \psi_{-}(0) \\
    \psi'_{+}(0) \\
    \psi'_{-}(0) \\
  \end{pmatrix}
\end{equation}
\begin{widetext}
\begin{equation}
  A = \frac{1}{4}
  \begin{pmatrix}
    e^{S_1 z} & e^{-S_1 z} & e^{S_3 z} & e^{-S_3 z} \\[0.3em]
    \mu e^{S_1 z} & \mu e^{-S_1 z} & -\mu e^{S_3 z} & -\mu e^{-S_3 z} \\[0.3em]
    S_1 e^{S_1 z} & -S_1 e^{-S_1 z} & S_3 e^{S_3 z} & -S_3e^{-S_3 z} \\[0.3em]
    \mu S_1 e^{S_1 z} & -\mu S_1 e^{-S_1 z} & -\mu S_3 e^{S_3 z} & \mu S_3 e^{-S_3 z} \\[0.3em]
  \end{pmatrix}
  \begin{pmatrix}
    1 & \frac{1}{\mu} & \frac{1}{S_1} & \frac{1}{\mu S_1} \\[0.3em]
    1 & \frac{1}{\mu} & \frac{-1}{S_1} & \frac{-1}{\mu S_1} \\[0.3em]
    1 & \frac{-1}{\mu} & \frac{1}{S_3} & \frac{-1}{\mu S_3} \\[0.3em]
    1 & \frac{-1}{\mu} & \frac{-1}{S_3} & \frac{1}{\mu S_3} \\[0.3em]
  \end{pmatrix}
\end{equation}

\begin{equation}
  A = \frac{1}{4}
  \left(
  \begin{matrix}
    1 & 1 & 1 & 1 \\[0.3em]
    \mu & \mu & -\mu & -\mu \\[0.3em]
    S_1 & -S_1 & S_3 & -S_3 \\[0.3em]
    \mu S_1  & -\mu S_1 & -\mu S_3 & \mu S_3 \\[0.3em]
  \end{matrix}
  \right)_{\!\!\!\!l}
  \left(
  \begin{matrix}
    e^{S_1 z} & 0 & 0 & 0 \\[0.3em]
    0 & e^{-S_1 z} & 0 & 0 \\[0.3em]
    0 & 0 & e^{S_3 z} & 0 \\[0.3em]
    0 & 0 & 0 & e^{-S_3 z} \\[0.3em]
  \end{matrix}
  \right)
  \begin{pmatrix}
    1 & \frac{1}{\mu} & \frac{1}{S_1} & \frac{1}{\mu S_1} \\[0.3em]
    1 & \frac{1}{\mu} & \frac{-1}{S_1} & \frac{-1}{\mu S_1} \\[0.3em]
    1 & \frac{-1}{\mu} & \frac{1}{S_3} & \frac{-1}{\mu S_3} \\[0.3em]
    1 & \frac{-1}{\mu} & \frac{-1}{S_3} & \frac{1}{\mu S_3} \\[0.3em]
  \end{pmatrix}
\end{equation}
\end{widetext}
which is valid for any translation $z$ over a region of constant scattering 
length density (and therefore $\mu$, $S_1$ and $S_3$), and is identical to the 
matrix in Eq. 128 of \cite{PNRMajkrzakChapter}.

Because the wavefunction and first derivatives are continuous over any boundary,
one can then use this transfer matrix to ``advance'' the wavefunction through
an arbitrary number of layers, recalculating the matrix at each interface so that
$S_1$, $S_3$ and $\mu$ are appropriate for the current layer.
\begin{equation}
  \displaystyle \Psi_l = \left( \prod_{l=N}^1 A_l \right) \Psi_0
\end{equation}

Using the additional boundary condition that
the incident wave only comes from one direction, i.e.
\begin{equation}
  C_{2,N} = C_{4,N} = D_{2,N} = D_{4,N} \equiv 0
\end{equation}
one can use this matrix product to calculate the reflectivity terms 
$r_{++}, r_{+-}, r_{-+}, r_{--}$.

\section{Construction of transfer matrix in terms of C coefficients}

It is possible to make an equivalent transfer matrix for the coefficients $C_j$, 
since they contain the same information as $\Psi$.

In order to do so we note that on either side of a boundary (between layer $l$
and layer $l+1$)
\begin{equation}
\Psi(z)_l = \Psi(z)_{l+1}
\end{equation}
and plugging into Eq. \ref{eq:cd_to_pz}
\begin{widetext}
\begin{equation}
  \left(
  \begin{matrix}
    e^{S_1 z} & e^{-S_1 z} & e^{S_3 z} & e^{-S_3 z} \\
    \mu_l e^{S_1 z} & \mu_l e^{-S_1 z} & -\mu_l e^{S_3 z} & -\mu_l e^{-S_3 z} \\
    S_1 e^{S_1 z} & -S_1 e^{-S_1 z} & S_3 e^{S_3 z} & -S_3e^{-S_3 z} \\
    \mu_l S_1 e^{S_1 z} & -\mu_l S_1 e^{-S_1 z} & -\mu_l S_3 e^{S_3 z} & \mu_l S_3 e^{-S_3 z} \\
  \end{matrix}
  \right)_{\!\!l}\!\!\!
  \left(
  \begin{matrix}
    C_{1} \\
    C_{2} \\
    C_{3} \\
    C_{4} \\
  \end{matrix}
  \right)_{\!\!l}
  \!\!=\!\!
  \left(
  \begin{matrix}
    e^{S_1 z} & e^{-S_1 z} & e^{S_3 z} & e^{-S_3 z} \\
    \mu_l e^{S_1 z} & \mu_l e^{-S_1 z} & -\mu_l e^{S_3 z} & -\mu_l e^{-S_3 z} \\
    S_1 e^{S_1 z} & -S_1 e^{-S_1 z} & S_3 e^{S_3 z} & -S_3e^{-S_3 z} \\
    \mu_l S_1 e^{S_1 z} & -\mu_l S_1 e^{-S_1 z} & -\mu_l S_3 e^{S_3 z} & \mu_l S_3 e^{-S_3 z} \\
  \end{matrix}
  \right)_{\!\!l+1}\!\!\!\!\!
  \left(
  \begin{matrix}
    C_{1} \\
    C_{2} \\
    C_{3} \\
    C_{4} \\
  \end{matrix}
  \right)_{\!\!l+1}
\end{equation}


%\begin{equation}
%  \left(
%  \begin{matrix}
%    1 & 1 & 1 & 1 \\[0.3em]
%    \mu & \mu & -\mu & -\mu \\[0.3em]
%    S_1 & -S_1 & S_3 & -S_3 \\[0.3em]
%    \mu S_1  & -\mu S_1 & -\mu S_3 & \mu S_3 \\[0.3em]
%  \end{matrix}
%  \right)_{\!\!\!\!\!l+1}\!\!\!\!
%  \left(
%  \begin{matrix}
%    e^{S_1 z} & 0 & 0 & 0 \\[0.3em]
%    0 & e^{-S_1 z} & 0 & 0 \\[0.3em]
%    0 & 0 & e^{S_3 z} & 0 \\[0.3em]
%    0 & 0 & 0 & e^{-S_3 z} \\[0.3em]
%  \end{matrix}
%  \right)_{\!\!\!\!l+1}\!\!\!\!\!
%  \left(
%  \begin{matrix}
%    C_{1} \\[0.3em]
%    C_{2} \\[0.3em]
%    C_{3} \\[0.3em]
%    C_{4} \\[0.3em]
%  \end{matrix}
%  \right)_{\!\!l+1}\!\!\!\!\!\!
%\end{equation}
%\begin{eqnarray}
%  \begin{pmatrix}
%    e^{S_{1,l} z} & e^{-S_{1,l} z} & e^{S_{3,l} z} & e^{-S_{3,l} z} \\
%    \mu_l e^{S_{1,l} z} & \mu_l e^{-S_{1,l} z} & -\mu_l e^{S_{3,l} z} & -\mu_l e^{-S_{3,l} z} \\
%    S_{1,l} e^{S_{1,l} z} & -S_{1,l} e^{-S_{1,l} z} & S_{3,l} e^{S_{3,l} z} & -S_{3,l}e^{-S_{3,l} z} \\
%    \mu_l S_{1,l} e^{S_{1,l} z} & -\mu_l S_{1,l} e^{-S_{1,l} z} & -\mu_l S_{3,l} e^{S_{3,l} z} & \mu_l S_{3,l} e^{-S_{3,l} z} \\
%  \end{pmatrix}
%  \begin{pmatrix}
%    C_{1,l} \\
%    C_{2,l} \\
%    C_{3,l} \\
%    C_{4,l} \\
%  \end{pmatrix}
%  =\nonumber \\
%  \begin{pmatrix}
%    e^{S_{1,l+1} z} & e^{-S_{1,l+1} z} & e^{S_{3,l+1} z} & e^{-S_{3,l+1} z} \\
%    \mu_{l+1} e^{S_{1,l+1} z} & \mu_{l+1} e^{-S_{1,l+1} z} & -\mu_{l+1} e^{S_{3,l+1} z} & -\mu_{l+1} e^{-S_{3,l+1} z} \\
%    S_{1,l+1} e^{S_{1,l+1} z} & -S_{1,l+1} e^{-S_{1,l+1} z} & S_{3,l+1} e^{S_{3,l+1} z} & -S_{3,l+1}e^{-S_{3,l+1} z} \\
%    \mu_{l+1} S_{1,l+1} e^{S_{1,l+1} z} & -\mu_{l+1} S_{1,l+1} e^{-S_{1,l+1} z} & -\mu_{l+1} S_{3,l+1} e^{S_{3,l+1} z} & \mu_{l+1} S_{3,l+1} e^{-S_{3,l+1} z} \\
%  \end{pmatrix}
%  \begin{pmatrix}
%    C_{1,l+1} \\
%    C_{2,l+1} \\
%    C_{3,l+1} \\
%    C_{4,l+1} \\
%  \end{pmatrix}
%\end{eqnarray}

\end{widetext}

Applying the inverse matrix of the RHS gives a new $4 \times 4$ matrix $B$ that
transfers $C_{j,l}$ to $C_{j,l+1}$.  
\begin{equation}
  \left(
  \begin{matrix}
    C_{1} \\
    C_{2} \\
    C_{3} \\
    C_{4} \\
  \end{matrix}
  \right)_{\!\!l+1}\!\!\!\!\!\!
  = B_l
  \left(
  \begin{matrix}
    C_{1} \\
    C_{2} \\
    C_{3} \\
    C_{4} \\
  \end{matrix}
  \right)_{\!\!l}
\end{equation}

\begin{widetext}
\begin{equation}
B_l=\frac{1}{4}
  \left(
  \begin{matrix}
    e^{-S_1 z} & 0 & 0 & 0 \\[0.3em]
    0 & e^{S_1 z} & 0 & 0 \\[0.3em]
    0 & 0 & e^{-S_3 z} & 0 \\[0.3em]
    0 & 0 & 0 & e^{S_3 z} \\[0.3em]
  \end{matrix}
  \right)_{\!\!\!\!l+1}\!\!\!
  \left(
  \begin{matrix}
    1 & \frac{1}{\mu} & \frac{1}{S_1} & \frac{1}{\mu S_1} \\[0.3em]
    1 & \frac{1}{\mu} & \frac{-1}{S_1} & \frac{-1}{\mu S_1} \\[0.3em]
    1 & \frac{-1}{\mu} & \frac{1}{S_3} & \frac{-1}{\mu S_3} \\[0.3em]
    1 & \frac{-1}{\mu} & \frac{-1}{S_3} & \frac{1}{\mu S_3} \\[0.3em]
  \end{matrix}
  \right)_{\!\!\!\!l+1}\!\!\!
  \left(
  \begin{matrix}
    1 & 1 & 1 & 1 \\[0.3em]
    \mu & \mu & -\mu & -\mu \\[0.3em]
    S_1 & -S_1 & S_3 & -S_3 \\[0.3em]
    \mu S_1  & -\mu S_1 & -\mu S_3 & \mu S_3 \\[0.3em]
  \end{matrix}
  \right)_{\!\!\!\!l}
  \left(
  \begin{matrix}
    e^{S_1 z} & 0 & 0 & 0 \\[0.3em]
    0 & e^{-S_1 z} & 0 & 0 \\[0.3em]
    0 & 0 & e^{S_3 z} & 0 \\[0.3em]
    0 & 0 & 0 & e^{-S_3 z} \\[0.3em]
  \end{matrix}
  \right)_{\!\!\!\!l}
\end{equation}
\end{widetext}

The components of $B$ are listed in Table \ref{B_components}.
While this approach doesn't reduce the complexity of
the calculation of $r$ and $t$, it does offer advantages in the case of DWBA problems;
no extra work is done converting between $\Psi$ and $C_j$ at every layer, 
and the DWBA terms depend on $\int_{z_l}^{z_{l+1}} \Psi(z)$ which is much easier to
calculate as $\int_{z_l}^{z_{l+1}} \sum_j C_{j,l} e^{S_{j,l} z}$ than 
$\left(\int_{z_l}^{z_{l+1}} A_l(z)\right) \Psi(z=z_l)$.


\begin{table}
\caption{Elements of the $B$-matrix, which transfers $C_{j,l} \rightarrow C_{j,l+1}$}

\begin{equation}
\begin{array}{l}
  B_{11} = \frac{1}{4} e^{(S_{1,l} - S_{1,l+1})z}\left( 1 + \frac{\mu_l}{\mu_{l+1}} + \frac{S_{1,l}}{S_{1,l+1}} + \frac{\mu_l S_{1,l}}{\mu_{l+1} S_{1,l+1}} \right)\\
  B_{12} = \frac{1}{4} e^{(-S_{1,l} - S_{1,l+1})z}\left( 1 + \frac{\mu_l}{\mu_{l+1}} - \frac{S_{1,l}}{S_{1,l+1}} - \frac{\mu_l S_{1,l}}{\mu_{l+1} S_{1,l+1}} \right)\\
  B_{13} = \frac{1}{4} e^{(S_{3,l} - S_{1,l+1})z}\left( 1 - \frac{\mu_l}{\mu_{l+1}} + \frac{S_{3,l}}{S_{1,l+1}} - \frac{\mu_l S_{3,l}}{\mu_{l+1} S_{1,l+1}} \right)\\
  B_{14} = \frac{1}{4} e^{(-S_{3,l} - S_{1,l+1})z}\left( 1 - \frac{\mu_l}{\mu_{l+1}} - \frac{S_{3,l}}{S_{1,l+1}} + \frac{\mu_l S_{3,l}}{\mu_{l+1} S_{1,l+1}} \right)\\[1em]
  
  B_{21} = \frac{1}{4} e^{(S_{1,l} + S_{1,l+1})z}\left( 1 + \frac{\mu_l}{\mu_{l+1}} - \frac{S_{1,l}}{S_{1,l+1}} - \frac{\mu_l S_{1,l}}{\mu_{l+1} S_{1,l+1}} \right)\\
  B_{22} = \frac{1}{4} e^{(-S_{1,l} + S_{1,l+1})z}\left( 1 + \frac{\mu_l}{\mu_{l+1}} + \frac{S_{1,l}}{S_{1,l+1}} + \frac{\mu_l S_{1,l}}{\mu_{l+1} S_{1,l+1}} \right)\\
  B_{23} = \frac{1}{4} e^{(S_{3,l} + S_{1,l+1})z}\left( 1 - \frac{\mu_l}{\mu_{l+1}} - \frac{S_{3,l}}{S_{1,l+1}} + \frac{\mu_l S_{3,l}}{\mu_{l+1} S_{1,l+1}} \right)\\
  B_{24} = \frac{1}{4} e^{(-S_{3,l} + S_{1,l+1})z}\left( 1 - \frac{\mu_l}{\mu_{l+1}} + \frac{S_{3,l}}{S_{1,l+1}} - \frac{\mu_l S_{3,l}}{\mu_{l+1} S_{1,l+1}} \right)\\[1em]
  
  B_{31} = \frac{1}{4} e^{(S_{1,l} - S_{3,l+1})z}\left( 1 - \frac{\mu_l}{\mu_{l+1}} + \frac{S_{1,l}}{S_{3,l+1}} - \frac{\mu_l S_{1,l}}{\mu_{l+1} S_{3,l+1}} \right)\\
  B_{32} = \frac{1}{4} e^{(-S_{1,l} - S_{3,l+1})z}\left( 1 - \frac{\mu_l}{\mu_{l+1}} - \frac{S_{1,l}}{S_{3,l+1}} + \frac{\mu_l S_{1,l}}{\mu_{l+1} S_{3,l+1}} \right)\\
  B_{33} = \frac{1}{4} e^{(S_{3,l} - S_{3,l+1})z}\left( 1 + \frac{\mu_l}{\mu_{l+1}} + \frac{S_{3,l}}{S_{3,l+1}} + \frac{\mu_l S_{3,l}}{\mu_{l+1} S_{3,l+1}} \right)\\
  B_{34} = \frac{1}{4} e^{(-S_{3,l} - S_{3,l+1})z}\left( 1 + \frac{\mu_l}{\mu_{l+1}} - \frac{S_{3,l}}{S_{3,l+1}} - \frac{\mu_l S_{3,l}}{\mu_{l+1} S_{3,l+1}} \right)\\[1em]
  
  B_{41} = \frac{1}{4} e^{(S_{1,l} + S_{3,l+1})z}\left( 1 - \frac{\mu_l}{\mu_{l+1}} - \frac{S_{1,l}}{S_{1,l+1}} + \frac{\mu_l S_{1,l}}{\mu_{l+1} S_{1,l+1}} \right)\\
  B_{42} = \frac{1}{4} e^{(-S_{1,l} + S_{3,l+1})z}\left( 1 - \frac{\mu_l}{\mu_{l+1}} + \frac{S_{1,l}}{S_{1,l+1}} - \frac{\mu_l S_{1,l}}{\mu_{l+1} S_{1,l+1}} \right)\\
  B_{43} = \frac{1}{4} e^{(S_{3,l} + S_{3,l+1})z}\left( 1 + \frac{\mu_l}{\mu_{l+1}} - \frac{S_{3,l}}{S_{1,l+1}} - \frac{\mu_l S_{3,l}}{\mu_{l+1} S_{1,l+1}} \right)\\
  B_{44} = \frac{1}{4} e^{(-S_{3,l} + S_{3,l+1})z}\left( 1 + \frac{\mu_l}{\mu_{l+1}} + \frac{S_{3,l}}{S_{1,l+1}} + \frac{\mu_l S_{3,l}}{\mu_{l+1} S_{1,l+1}} \right)\\[1em]
  
\end{array}\nonumber
\end{equation}
\label{B_components}
\end{table}

The calculation of $r$ can by setting 
the incident polarization to $\pm 1$ and using the equations
\begin{equation}
  \begin{array}{l}
    C_{2,N} = B_{21} C_{1,0} + B_{22} C_{2,0} + B_{23} C_{3,0} + B_{24} C_{4,0} \equiv 0\\
    C_{4,N} = B_{41} C_{1,0} + B_{42} C_{2,0} + B_{43} C_{3,0} + B_{44} C_{4,0} \equiv 0    
  \end{array}
\end{equation}
noting that
\begin{equation}
  \begin{array}{l}
    C_{1,0} \equiv I_+ \\
    C_{2,0} \equiv r_+ \\
    C_{3,0} \equiv I_- \\
    C_{4,0} \equiv r_- \\
  \end{array}
\end{equation}
giving expressions for $r$
\begin{equation}
  \begin{array}{l}
    r_{++} = \frac{B_{24}B_{41} - B_{21}B_{44}}{B_{44}B_{22} - B_{24}B_{42}}\\[0.3em]
    r_{+-} = \frac{B_{21}B_{42} - B_{41}B_{22}}{B_{44}B_{22} - B_{42}B_{24}}\\[0.3em]
    r_{-+} = \frac{B_{24}B_{43} - B_{23}B_{44}}{B_{44}B_{22} - B_{24}B_{42}}\\[0.3em]
    r_{--} = \frac{B_{23}B_{42} - B_{43}B_{22}}{B_{44}B_{22} - B_{42}B_{24}}\\[0.3em]
  \end{array}
\end{equation}
then the transmission can be calculated, since 
we now know all the $C_{j,0}$, and 
\begin{equation}
  \begin{array}{l}
    t_+ \equiv C_{1,N} \\
    t_- \equiv C_{3,N} \\
  \end{array}
\end{equation}


\section{Rotation of reference frame}

The equations in the previous section were derived in a single reference frame, 
which for the sample is defined by convention to have $\hat z_{\textrm{SAM}} \parallel \vec Q$.
Given that the $\hat z$ direction is also by convention the quantization axis 
for spin, we have to apply a transformation if the lab quantization axis 
$(\hat z_{\textrm {LAB}} \parallel \vec H_{\textrm{guide}})$ is not collinear with
$\hat z_{\textrm{SAM}}$.

In order to mate the laboratory reference frame to the sample one, we have to do
a rotation.  As per Eq. 3.2.46 in \cite{SakuraiQuantum1994}, a spinor is rotated about
an axis according to
\begin{equation}
  \chi \rightarrow \exp\left(\frac{-i\boldsymbol{\sigma} \cdot \hat{\mathbf{n}} \phi}{2}\right) \chi
\end{equation} 

The rotation matrix around the $\hat x$-axis is (as in \cite{PNRMajkrzakChapter})
\begin{equation}
  U_R = 
  \begin{pmatrix}
    \cos(\epsilon/2) & i \sin(\epsilon/2) & 0 & 0 \\
    i \sin(\epsilon/2) & \cos(\epsilon/2) & 0 & 0 \\
    0 & 0 & \cos(\epsilon/2) & i \sin(\epsilon/2) \\
    0 & 0 & i \sin(\epsilon/2) & \cos(\epsilon/2) \\
  \end{pmatrix}
\end{equation}

The wavefunction in the incident medium has no coupling between
$\psi_+$ and $\psi_-$ when $\rho_{+-}, \rho_{-+} = 0$.  In practice, we take
the wavefunction in the incident medium and rotate it from the lab frame to
the sample frame, and then just inside the first layer of the sample we can calculate
the $C_j$.

\begin{equation}
  \begin{array}{l}
  U_R^{-1} \Psi_{\textrm{LAB}} = \Psi_{\textrm{SAM}} \\
  \chi^{-1} U_R^{-1} \Psi_{\textrm{LAB}} = \chi^{-1} \Psi_{\textrm{SAM}} = \bigg( C_j \bigg) \\
  \end{array}
\end{equation}

If one wishes to calculate the wavefunction in the laboratory frame throughout 
the sample, within each layer the solutions for $C_j$ can transferred to the lab
frame by noting that 

\begin{equation}
  \begin{array}{l}
    \left(
    \begin{matrix}
    \psi_+(z) \\
    \psi_-(z)
    \end{matrix}
    \right)_{\textrm{LAB}}
    = U_R
    \left(
    \begin{matrix}
    \psi_+(z) \\
    \psi_-(z)
    \end{matrix}
    \right)_{\textrm{SAM}}
    \\[2em]
    = \left(
    \begin{matrix}
      \cos(\epsilon/2)[\sum_j C_j e^{S_j z}] + i \sin(\epsilon/2)[\sum_j \mu_j C_j e^{S_j z}] \\
      i \sin(\epsilon/2)[\sum_j C_j e^{S_j z}] + \cos(\epsilon/2)[\sum_j \mu_j C_j e^{S_j z}] \\
    \end{matrix}
    \right)
  \end{array}
\end{equation}
and collecting terms that all have the same $e^{S_j z}$ dependence, we get
\begin{equation}
  \begin{array}{l}
    C^{\uparrow}_{j,\textrm{LAB}} = \left(\cos(\epsilon/2) + i\mu_j \sin(\epsilon/2)\right)C_{j,\textrm{SAM}} \\
    C^{\downarrow}_{j,\textrm{LAB}} = \left(\mu_j \cos(\epsilon/2) + i \sin(\epsilon/2)\right)C_{j,\textrm{SAM}} \\
  \end{array}
\end{equation}

Within the DWBA layer-by-layer calculations, of course, the reference frame doesn't
matter as long as $\langle \psi_f |$ and $| \psi_i \rangle$ are in the same 
reference frame, and it will be easiest to use the $\bigg(C_j\bigg)$ in the sample
frame.


%\begin{equation}
%\end{equation}

%\begin{thebibliography}{1}
%\bibitem{Dura_unpublished}J. Dura, private communication
%\end{thebibliography}
%%%
%\begin{equation}
%  \begin{pmatrix}
%    \psi_{+}(z) \\
%    \psi_{-}(z) \\
%    \psi'_{+}(z) \\
%    \psi'_{-}(z) \\
%  \end{pmatrix}
%  =
%  \begin{pmatrix}
%    1 & 1 & 1 & 1 \\
%    \mu & \mu & -\mu & -\mu \\
%    S_1 & -S_1 & S_3 & -S_3 \\
%    \mu S_1  & -\mu S_1 & -\mu S_3 & \mu S_3 \\
%  \end{pmatrix}
%  \begin{pmatrix}
%    C_1 e^{+S_1 z} \\
%    C_2 e^{-S_1 z} \\
%    C_3 e^{+S_3 z} \\
%    C_4 e^{-S_3 z} \\
%  \end{pmatrix}
%\end{equation}
%%%
\bibliography{/home/bbm/Documents/master}

\end{document}
